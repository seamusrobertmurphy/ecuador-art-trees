% Options for packages loaded elsewhere
\PassOptionsToPackage{unicode}{hyperref}
\PassOptionsToPackage{hyphens}{url}
\PassOptionsToPackage{dvipsnames,svgnames,x11names}{xcolor}
%
\documentclass[
  letterpaper,
  DIV=11,
  numbers=noendperiod]{scrartcl}

\usepackage{amsmath,amssymb}
\usepackage{iftex}
\ifPDFTeX
  \usepackage[T1]{fontenc}
  \usepackage[utf8]{inputenc}
  \usepackage{textcomp} % provide euro and other symbols
\else % if luatex or xetex
  \usepackage{unicode-math}
  \defaultfontfeatures{Scale=MatchLowercase}
  \defaultfontfeatures[\rmfamily]{Ligatures=TeX,Scale=1}
\fi
\usepackage{lmodern}
\ifPDFTeX\else  
    % xetex/luatex font selection
\fi
% Use upquote if available, for straight quotes in verbatim environments
\IfFileExists{upquote.sty}{\usepackage{upquote}}{}
\IfFileExists{microtype.sty}{% use microtype if available
  \usepackage[]{microtype}
  \UseMicrotypeSet[protrusion]{basicmath} % disable protrusion for tt fonts
}{}
\makeatletter
\@ifundefined{KOMAClassName}{% if non-KOMA class
  \IfFileExists{parskip.sty}{%
    \usepackage{parskip}
  }{% else
    \setlength{\parindent}{0pt}
    \setlength{\parskip}{6pt plus 2pt minus 1pt}}
}{% if KOMA class
  \KOMAoptions{parskip=half}}
\makeatother
\usepackage{xcolor}
\setlength{\emergencystretch}{3em} % prevent overfull lines
\setcounter{secnumdepth}{5}
% Make \paragraph and \subparagraph free-standing
\makeatletter
\ifx\paragraph\undefined\else
  \let\oldparagraph\paragraph
  \renewcommand{\paragraph}{
    \@ifstar
      \xxxParagraphStar
      \xxxParagraphNoStar
  }
  \newcommand{\xxxParagraphStar}[1]{\oldparagraph*{#1}\mbox{}}
  \newcommand{\xxxParagraphNoStar}[1]{\oldparagraph{#1}\mbox{}}
\fi
\ifx\subparagraph\undefined\else
  \let\oldsubparagraph\subparagraph
  \renewcommand{\subparagraph}{
    \@ifstar
      \xxxSubParagraphStar
      \xxxSubParagraphNoStar
  }
  \newcommand{\xxxSubParagraphStar}[1]{\oldsubparagraph*{#1}\mbox{}}
  \newcommand{\xxxSubParagraphNoStar}[1]{\oldsubparagraph{#1}\mbox{}}
\fi
\makeatother


\providecommand{\tightlist}{%
  \setlength{\itemsep}{0pt}\setlength{\parskip}{0pt}}\usepackage{longtable,booktabs,array}
\usepackage{calc} % for calculating minipage widths
% Correct order of tables after \paragraph or \subparagraph
\usepackage{etoolbox}
\makeatletter
\patchcmd\longtable{\par}{\if@noskipsec\mbox{}\fi\par}{}{}
\makeatother
% Allow footnotes in longtable head/foot
\IfFileExists{footnotehyper.sty}{\usepackage{footnotehyper}}{\usepackage{footnote}}
\makesavenoteenv{longtable}
\usepackage{graphicx}
\makeatletter
\newsavebox\pandoc@box
\newcommand*\pandocbounded[1]{% scales image to fit in text height/width
  \sbox\pandoc@box{#1}%
  \Gscale@div\@tempa{\textheight}{\dimexpr\ht\pandoc@box+\dp\pandoc@box\relax}%
  \Gscale@div\@tempb{\linewidth}{\wd\pandoc@box}%
  \ifdim\@tempb\p@<\@tempa\p@\let\@tempa\@tempb\fi% select the smaller of both
  \ifdim\@tempa\p@<\p@\scalebox{\@tempa}{\usebox\pandoc@box}%
  \else\usebox{\pandoc@box}%
  \fi%
}
% Set default figure placement to htbp
\def\fps@figure{htbp}
\makeatother
% definitions for citeproc citations
\NewDocumentCommand\citeproctext{}{}
\NewDocumentCommand\citeproc{mm}{%
  \begingroup\def\citeproctext{#2}\cite{#1}\endgroup}
\makeatletter
 % allow citations to break across lines
 \let\@cite@ofmt\@firstofone
 % avoid brackets around text for \cite:
 \def\@biblabel#1{}
 \def\@cite#1#2{{#1\if@tempswa , #2\fi}}
\makeatother
\newlength{\cslhangindent}
\setlength{\cslhangindent}{1.5em}
\newlength{\csllabelwidth}
\setlength{\csllabelwidth}{3em}
\newenvironment{CSLReferences}[2] % #1 hanging-indent, #2 entry-spacing
 {\begin{list}{}{%
  \setlength{\itemindent}{0pt}
  \setlength{\leftmargin}{0pt}
  \setlength{\parsep}{0pt}
  % turn on hanging indent if param 1 is 1
  \ifodd #1
   \setlength{\leftmargin}{\cslhangindent}
   \setlength{\itemindent}{-1\cslhangindent}
  \fi
  % set entry spacing
  \setlength{\itemsep}{#2\baselineskip}}}
 {\end{list}}
\usepackage{calc}
\newcommand{\CSLBlock}[1]{\hfill\break\parbox[t]{\linewidth}{\strut\ignorespaces#1\strut}}
\newcommand{\CSLLeftMargin}[1]{\parbox[t]{\csllabelwidth}{\strut#1\strut}}
\newcommand{\CSLRightInline}[1]{\parbox[t]{\linewidth - \csllabelwidth}{\strut#1\strut}}
\newcommand{\CSLIndent}[1]{\hspace{\cslhangindent}#1}

\KOMAoption{captions}{tableheading}
\makeatletter
\@ifpackageloaded{caption}{}{\usepackage{caption}}
\AtBeginDocument{%
\ifdefined\contentsname
  \renewcommand*\contentsname{Table of contents}
\else
  \newcommand\contentsname{Table of contents}
\fi
\ifdefined\listfigurename
  \renewcommand*\listfigurename{List of Figures}
\else
  \newcommand\listfigurename{List of Figures}
\fi
\ifdefined\listtablename
  \renewcommand*\listtablename{List of Tables}
\else
  \newcommand\listtablename{List of Tables}
\fi
\ifdefined\figurename
  \renewcommand*\figurename{Figure}
\else
  \newcommand\figurename{Figure}
\fi
\ifdefined\tablename
  \renewcommand*\tablename{Table}
\else
  \newcommand\tablename{Table}
\fi
}
\@ifpackageloaded{float}{}{\usepackage{float}}
\floatstyle{ruled}
\@ifundefined{c@chapter}{\newfloat{codelisting}{h}{lop}}{\newfloat{codelisting}{h}{lop}[chapter]}
\floatname{codelisting}{Listing}
\newcommand*\listoflistings{\listof{codelisting}{List of Listings}}
\makeatother
\makeatletter
\makeatother
\makeatletter
\@ifpackageloaded{caption}{}{\usepackage{caption}}
\@ifpackageloaded{subcaption}{}{\usepackage{subcaption}}
\makeatother

\usepackage{bookmark}

\IfFileExists{xurl.sty}{\usepackage{xurl}}{} % add URL line breaks if available
\urlstyle{same} % disable monospaced font for URLs
\hypersetup{
  pdftitle={Addressing the 2020 Emissions Reversal},
  colorlinks=true,
  linkcolor={blue},
  filecolor={Maroon},
  citecolor={Blue},
  urlcolor={Blue},
  pdfcreator={LaTeX via pandoc}}


\title{Addressing the 2020 Emissions Reversal}
\usepackage{etoolbox}
\makeatletter
\providecommand{\subtitle}[1]{% add subtitle to \maketitle
  \apptocmd{\@title}{\par {\large #1 \par}}{}{}
}
\makeatother
\subtitle{Credit Reversal Review, Buffer Pool Adjustments \& Management
Strategies for Ecuador's ART TREES Program}
\author{}
\date{2024-12-17}

\begin{document}
\maketitle

\renewcommand*\contentsname{Table of contents}
{
\hypersetup{linkcolor=}
\setcounter{tocdepth}{3}
\tableofcontents
}

\hypertarget{introduction}{%
\section{1. Introduction}\label{introduction}}

This report addresses the 2020 emissions reversal within Ecuador's REDD+
program under the Architecture for REDD+ Transactions (ART) TREES
Standard Version 2.0 (ART 2021). It outlines the required compensatory
actions, the documented 2020 reversal event, strategies for credit
management, and recommendations for upcoming validation audits.

Under Section 7.1 of the TREES Standard V2.0, reversals are defined as
emissions exceeding the established baseline during the crediting
period, which must be compensated by the deduction of credits verified
from other years within the same crediting period (Section 7.1.3). In
addition, a reassessment of non-permanence risks may result in an
increase in the buffer pool contribution rate for future crediting
periods\hspace{0pt} (Section 7.1.2).

Given the material impact of the 2020 reversal, verifiers are expected
to apply heightened scrutiny during the upcoming validation and
verification processes. To ensure that Ecuador's TMR and TRD submissions
align with ART requirements, secure validation, and minimize the
financial implications of reversal compensation, this report offers
targeted recommendations and to strengthen compliance and program
resilience.

\hypertarget{objectives}{%
\section{2. Objectives}\label{objectives}}

Specifically, this report aligns with deliverables defined in the
contrat's terms of reference and focuses on the following objectives:

\textbf{Analysis of the 2020 Emissions Reversal}

\begin{itemize}
\tightlist
\item
  Review the reversal loss calculation to confirm compliance with TREES
  Sections 5, 7, and 10 regarding crediting levels and emissions
  reductions.
\item
  Assess the impact of the reversal on future crediting periods,
  including its effect on credit availability and buffer pool
  contributions.
\item
  Provide recommendations for managing the reversal, including
  appropriate credit deductions and mitigation strategies to prevent
  similar events in future periods.
\end{itemize}

\textbf{Guidance on Buffer Pool Adjustments \& Credit Allocations}

\begin{itemize}
\tightlist
\item
  Develop strategies to address buffer pool contributions post-reversal,
  including reassessment of non-permanence risks in alignment with TREES
  Section 7.1.3.
\item
  Provide technical advice on credit allocation strategies.
\item
  Ensure transparent documentation of credit adjustments, with robust
  justifications, for inclusion in the next TREES Monitoring Report
  (TMR).
\end{itemize}

\textbf{Strategies for Mitigating Future Reversal Risks}

\begin{itemize}
\tightlist
\item
  Conduct review of Ecuador's National Forest Monitoring System (NFMS)
  and MRV processes to confirm alignment with TREES requirements,
  including components of:

  \begin{itemize}
  \tightlist
  \item
    QA/QC protocols for Activity Data (AD) and Emission Factors (EF).
  \item
    Ensuring EF confidence intervals meet TREES standards.
  \end{itemize}
\item
  Recommend enhanced remote sensing technologies and early-warning
  systems to detect and address annual or seasonal deforestation risks.
\item
  Recommend targeted scaling of deforestation mitigation programs,
  including through community-based forest management initiatives,
  reforestation and restoration projects, and integration of economic
  incentives for sustainable agricultural practices to reduce forest
  pressures.
\end{itemize}

\textbf{Technical Preparedness for Upcoming Validation Audit}

\begin{itemize}
\item
  Build capacity through training and technical support to prepare for
  heightened scrutiny during validation and verification audits,
  including recommendations for:

  \begin{itemize}
  \item
    Responding effectively to Validation and Verification Body (VVB)
    findings.
  \item
    Ensuring consistency and compliance across reporting periods and
    submissions.
  \end{itemize}
\end{itemize}

\hypertarget{findings}{%
\section{\texorpdfstring{3.
\textbf{Findings}}{3. Findings}}\label{findings}}

\hypertarget{analysis-of-the-2020-reversal}{%
\subsection{3.1 Analysis of the 2020
Reversal}\label{analysis-of-the-2020-reversal}}

Under TREES Section 7.1, reversals occur when emissions exceed the
established baseline during the crediting period, requiring compensation
through credit deductions from the same period. In 2020, Ecuador's
reported emissions (81,756,832.52 tCO₂-e) exceeded the baseline
emissions (69,394,984.52 tCO₂-e) by 12,361,848 tCO₂-e, constituting a
significant reversal\hspace{0pt}\hspace{0pt}.

\begin{longtable}[]{@{}
  >{\centering\arraybackslash}p{(\columnwidth - 0\tabcolsep) * \real{1.0000}}@{}}
\toprule\noalign{}
\endhead
\bottomrule\noalign{}
\endlastfoot
\emph{Reversal Loss = 2020 Emissions -- Baseline Annual Emissions} \\
\emph{Reversal Loss = 81,756,832.52 - 69,394,984.52 = 12,361,848
tCO\textsubscript{2}\textsuperscript{-e}} \\
\end{longtable}

The reversal necessitates compensatory actions to ensure compliance with
TREES requirements.

\hypertarget{guidance-on-buffer-pool-adjustments}{%
\subsection{\texorpdfstring{\textbf{3.2 Guidance on Buffer Pool
Adjustments}}{3.2 Guidance on Buffer Pool Adjustments}}\label{guidance-on-buffer-pool-adjustments}}

The TREES Standards V2.0 emphasizes the importance of a robust buffer
pool to address non-permanence risks and ensure the permanence of
emission reductions. To address the 2020 reversal, Ecuador must take the
following actions:

\textbf{Mandatory Increase in Buffer Pool Contribution:}\\
As outlined in TREES Section 7.1.3, the 2020 reversal triggers a
mandatory 5\% increase in buffer pool contributions for a period of five
years. This ensures that the buffer pool compensates for the elevated
non-permanence risks associated with the event. During this period, the
adjusted buffer pool rate must apply to all future credit issuance.

\textbf{Mitigating Factors for Buffer Rate Adjustment:}\\
In Section 7.1.1 of the TREES Standards, requirements allow
jurisdictions to lower their baseline buffer contribution rate of 25\%
if specific mitigating factors are demonstrated. Based on the activities
detailed in Ecuador's TRD, the program may qualify for Mitigating Factor
3 (-5\%) due to its national reversal mitigation actions aligned with
Cancun Safeguard F, which include:

\begin{itemize}
\tightlist
\item
  National Forest Monitoring System (NFMS): Tracks emissions, monitors
  leakage, and supports data-driven mitigation strategies.
\item
  Criminal Code Enforcement: Provisions to address environmental crimes
  such as land invasion and illegal logging.
\item
  Results-Based Payment Initiatives: Programs like the REM Ecuador
  initiative and GCF-supported activities promote sustainable land use
  and reversal risk mitigation.
\item
  Financial and Institutional Capacity Building: Measures like the
  Environmental and Social Risk Analysis System (SARAS) integrate
  reversal mitigation into financial strategies and support sustainable
  land use practices.
\end{itemize}

Given these activities, Ecuador's adjusted buffer contribution rate
would likely be reduced from 25\% to \textbf{20\%} (25\% base rate minus
5\% for Mitigating Factor 3; TREES Standards V2.0: 42).

\textbf{Addressing Deficits in Buffer Contributions:}\\
If the credits retired to compensate for the 2020 reversal exceed
Ecuador's cumulative buffer contributions, the shortfall must be
replenished. This can be achieved by:

\begin{itemize}
\tightlist
\item
  Allocating future credits to the buffer pool until the deficit is
  resolved.
\item
  Ensuring transparent documentation in the next TREES Monitoring Report
  to demonstrate compliance with Section 7.1.3.
\end{itemize}

\hypertarget{guidance-on-credit-allocation-strategies}{%
\subsection{\texorpdfstring{\textbf{3.3 Guidance on Credit Allocation
Strategies}}{3.3 Guidance on Credit Allocation Strategies}}\label{guidance-on-credit-allocation-strategies}}

To minimize the financial impact of the 2020 reversal and maximize the
value of Ecuador's remaining credits, the following credit allocation
strategies are recommended:

\textbf{Prioritize Older, Lower-Value Vintages for Reversal
Compensation:}\\
Use credits issued in 2017 and 2018 to offset the reversal. These older
vintages typically hold lower market value due to depreciation over time
and are ideal for compliance purposes.

\textbf{Preserve Recent, Higher-Value Credits for Future
Transactions:}\\
Retain credits from 2019 and 2021, which are likely to command higher
prices in voluntary and compliance carbon markets. Preserving these
credits positions Ecuador to maximize revenue from future transactions
and maintain flexibility in managing carbon assets.

\textbf{Ensure Transparent Documentation:}\\
Provide clear and detailed records of all credit adjustments in the next
TREES Monitoring Report, including:

\begin{itemize}
\tightlist
\item
  The volume of credits retired and the vintages used.
\item
  Justifications for the prioritization of certain vintages.
\item
  Evidence of buffer pool replenishment, if applicable.
\end{itemize}

\hypertarget{mitigating-risk-of-future-reversals}{%
\subsection{3.4 Mitigating Risk of Future
Reversals}\label{mitigating-risk-of-future-reversals}}

Drivers of Reversal:\\
In addition to population pressures and labour markets, the economic
downturn caused by the COVID-19 pandemic likely contributed to an
increased reliance on forest resources for subsistence and informal
activities. As referenced in the TMR and TRD, key drivers of
deforestation in the country include the following:

\begin{itemize}
\item
  Agricultural expansion: Expansion of agricultural land involving
  slash-and-burn practices, particularly in coastal regions where
  tradition of similar land-use decisions have contributed to
  deforestation (TMR, p.14; TRD p.17, p.76)
\item
  Forest management: Selective logging and resource extraction for
  fuelwood and timber contributing to deforestation in the Andean region
  (Sierra, Calva, and Guevara 2021).
\item
  Population pressures: Urban to rural migration has exacerbated
  land-use change in Andean forests.
\item
  Limited Enforcement: Conservation and forest management policies were
  less enforceable during external shocks, particularly due to the
  COVID-19 pandemic.
\end{itemize}

Mitigation of Reversal:\\
Section 6.1 requires\ldots{}

\hypertarget{technical-preparedness-for-upcoming-validation-audit}{%
\subsection{\texorpdfstring{\textbf{3.5 Technical Preparedness for
Upcoming Validation
Audit}}{3.5 Technical Preparedness for Upcoming Validation Audit}}\label{technical-preparedness-for-upcoming-validation-audit}}

The need for enhanced monitoring, including the application of advanced
remote sensing technologies and early-warning systems, is supported in
several key aspects of the TREES Standards. Ecuador may focus on
strengthening its monitoring, reporting, and verification (MRV) systems,
improving technical capabilities. Specifically, Ecuador must reinforce
its National Forest Monitoring System (NFMS) and MRV processes to
improve detection and response to deforestation risks, as stipulated in
the following requirements of the TREES Standards V2.0.

Monitoring Plan:\\
Section 6.1 requires participants to develop a robust monitoring plan as
part of their TREES Registration Document. The monitoring plan must
detail the parameters monitored, the frequency and methods of data
collection, and the responsible parties. It also mandates quality
control checks and internal data quality assurance
procedures\hspace{0pt}\hspace{0pt}.

QA/QC Protocols:\\
Section 4.1 requires jurisdictions to apply IPCC Tier 2 or Tier 3
approaches for AD and EF, ensuring accurate and transparent emissions
reporting. The section also mandates the inclusion of detailed
descriptions, such as methodological protocols, standard operating
procedures (SOPs), and quality assurance/quality control procedures, in
the TREES Monitoring Report\hspace{0pt}\hspace{0pt}.

\hypertarget{annexes}{%
\subsubsection{Annexes}\label{annexes}}

\begin{itemize}
\item
  \textbf{Annex 1}: Detailed Reversal Compensation Calculations.
\item
  \textbf{Annex 2}: Buffer Pool Risk Assessment Matrix.
\item
  \textbf{Annex 3}: Maps of Deforestation and Emissions Hotspots for
  2020.
\item
  \textbf{Annex 4}: Glossary of Terms.
\end{itemize}

\hypertarget{refs}{}
\begin{CSLReferences}{1}{0}
\leavevmode\vadjust pre{\hypertarget{ref-artREDDEnvironmentalExcellence2021}{}}%
ART, Secretariat. 2021. \emph{The {REDD}+ {Environmental Excellence
Standard}} (version 2.0).
\url{https://www.artredd.org/wp-content/uploads/2021/12/TREES-2.0-August-2021-Clean.pdf}.

\leavevmode\vadjust pre{\hypertarget{ref-book}{}}%
Sierra, Rodrigo, Oscar Calva, and A. Guevara. 2021. \emph{La
{Deforestación} En El {Ecuador}, 1990-2018. {Factores} Promotores y
Tendencias Recientes.}

\end{CSLReferences}




\end{document}
